\documentclass[handout,nooutcomes]{ximera}
%handout
%wordchoicegiven
%space
%nooutcomes
\title{Math 160 Lab 2}
\author{Ben Sencindiver} %Used Bart Snapp and Jim Fowler's mooculus textbook as a guide
%irrelevantchange
%\usepackage{todonotes}

\newcommand{\todo}{}

\usepackage{tkz-euclide}
\tikzset{>=stealth} %% cool arrow head
\tikzset{shorten <>/.style={ shorten >=#1, shorten <=#1 } } %% allows shorter vectors

\usetikzlibrary{backgrounds} %% for boxes around graphs
\usetikzlibrary{shapes,positioning}  %% Clouds and stars
\usetikzlibrary{matrix} %% for matrix
\usetkzobj{all}
\usepackage[makeroom]{cancel} %% for strike outs
%\usepackage{mathtools} %% for pretty underbrace % Breaks Ximera
\usepackage{multicol}


\newcommand{\RR}{\mathbb R}
%\renewcommand{\d}{\,d\!}
\renewcommand{\d}{\mathop{}\!d\!}
\newcommand{\dd}[2][]{\frac{\d #1}{\d #2}}
\renewcommand{\l}{\ell}
\newcommand{\ddx}{\frac{d}{\d x}}

\newcommand{\zeroOverZero}{\ensuremath{\boldsymbol{\tfrac{0}{0}}}}
\newcommand{\inftyOverInfty}{\ensuremath{\boldsymbol{\tfrac{\infty}{\infty}}}}
\newcommand{\zeroOverInfty}{\ensuremath{\boldsymbol{\tfrac{0}{\infty}}}}
\newcommand{\zeroTimesInfty}{\ensuremath{\small\boldsymbol{0\cdot \infty}}}
\newcommand{\inftyMinusInfty}{\ensuremath{\small\boldsymbol{\infty - \infty}}}
\newcommand{\oneToInfty}{\ensuremath{\boldsymbol{1^\infty}}}
\newcommand{\zeroToZero}{\ensuremath{\boldsymbol{0^0}}}
\newcommand{\inftyToZero}{\ensuremath{\boldsymbol{\infty^0}}}


\newcommand{\numOverZero}{\ensuremath{\boldsymbol{\tfrac{\#}{0}}}}
\newcommand{\dfn}{\textbf}
%\newcommand{\unit}{\,\mathrm}
\newcommand{\unit}{\mathop{}\!\mathrm}
\newcommand{\eval}[1]{\bigg[ #1 \bigg]}
\renewcommand{\epsilon}{\varepsilon}
\renewcommand{\iff}{\Leftrightarrow}

\DeclareMathOperator{\arccot}{arccot}
\DeclareMathOperator{\arcsec}{arcsec}
\DeclareMathOperator{\arccsc}{arccsc}
\DeclareMathOperator{\si}{Si}


\colorlet{textColor}{black} 
\colorlet{background}{white}
\colorlet{penColor}{blue!50!black} % Color of a curve in a plot
\colorlet{penColor2}{red!50!black}% Color of a curve in a plot
\colorlet{penColor3}{red!50!blue} % Color of a curve in a plot
\colorlet{penColor4}{green!50!black} % Color of a curve in a plot
\colorlet{penColor5}{orange!80!black} % Color of a curve in a plot
\colorlet{fill1}{penColor!20} % Color of fill in a plot
\colorlet{fill2}{penColor2!20} % Color of fill in a plot
\colorlet{fillp}{fill1} % Color of positive area
\colorlet{filln}{penColor2!20} % Color of negative area
\colorlet{fill3}{penColor3!20} % Fill
\colorlet{fill4}{penColor4!20} % Fill
\colorlet{fill5}{penColor5!20} % Fill
\colorlet{gridColor}{gray!50} % Color of grid in a plot

\pgfmathdeclarefunction{gauss}{2}{% gives gaussian
  \pgfmathparse{1/(#2*sqrt(2*pi))*exp(-((x-#1)^2)/(2*#2^2))}%
}



\newcommand{\fullwidth}{}
\newcommand{\normalwidth}{}


%% makes a snazzy t-chart for evaluating functions
%\newenvironment{tchart}{\rowcolors{2}{}{background!90!textColor}\array}{\endarray}

%%This is to help with formatting on future title pages.
\newenvironment{sectionOutcomes}{}{} 


\outcome{Define linear approximation as an application of the tangent to a curve.}
\outcome{Find the linear approximation to a function at a point and use it to approximate the function value.}
\outcome{Identify when a linear approximation can be used.}
\outcome{Label a graph with the appropriate quantities used in linear approximation.}
\outcome{Find the error of a linear approximation.}

\begin{document}

\section{Math 160 Lab 2 \\ Linear Approximation}

\begin{abstract}
This is Lab 2 for Math 160 - Due Wednesday, 12 October, 2016. This lab will cover tolerance and provide an introduction to the idea of limits. Unless stated otherwise, compute all values to $6$ decimal places.

We use a method called ``linear approximation'' to estimate the value of (complicated) functions at a given point.
\end{abstract}

\maketitle

%\begin{example}
%Suppose you want to approximate the function $f(x) = (x-2)^2  + 4$ at $x=3$.
%
%Given a line with slope $m$ through the point $(a, b)$, what is the point-slope form of the line?
%\[
%y = \answer[given]{m(x - a) + b}
%\]
%
%What is the slope of the line tangent to $f(x)$ at $x=3$?
%\[
%\answer{2}
%\]
%
%\begin{problem}
%What point should an approximation of $f(x)$ at $x=3$ go through?
%
%\[
%(\answer[given]{3}, \answer[given]{5})
%\]
%\begin{hint}
%Should the linear approximation go through the point at which it is approximating?
%\end{hint}
%\end{problem}
%
%\begin{problem}
%What is the equation of the line tangent to $f(x)$ at $x-2$?
%\[
%y = \answer[given]{2}x + \answer[given]{-1}
%\]
%\begin{hint}
%Think $y=mx+b$
%\end{hint}
%
%\begin{feedback}
%Your line should look like this.
%\[
%\graph{f(x) = (x-2)^2+4, y = 2(x-3) + 5}
%\]
%\end{feedback}
%\end{problem}
%
%
%
%
%%\section{Linear approximation}
%Given a function, a \textit{linear approximation} (at $x=x_0$) is a fancy phrase
%for something you already know:
%\begin{center}
%\begin{quote}
%  \textbf{The line tangent to the function (at $x=x_0$).}
%\end{quote}
%\end{center}
%
%
%\begin{definition}\index{linear approximation}
%If $f$ is a differentiable function at $x=a$, then a \textbf{linear
%  approximation} for $f$ at $x=a$ is given by
%\[
%\l(x) = f'(a)(x-a) +f(a).
%\]
%\end{definition}
%
%
%Note that $\l(x)$ is just the line tangent to $f(x)$ at $x=a$.
%
%
%\begin{example}
%Below is the graph of $f(x) = 1/8x^3 + (3/8)x^2 -3/4 - 2$ and the line
%tangent to $f(x)$ at $x=a$. Feel free to explore how the tangent line
%(in purple) changes as $a$ varies by using the slider. You may need to
%hit `-' (zoom out) in order to see the linear approximations for large $a$ values.
%\[
%\graph[panel]{a=1, f(x) = 1/8x^3 + (3/8)x^2 -3/4 - 2, t(x) = f'(a)(x-a)+ f(a)}
%\]
%
%Notice how the linear approximation (the purple line) hugs
%$f(x)$ (the green line) at the $x=a$ value!
%\end{example}
%
%A linear approximation of $f$ is a ``good'' approximation as long as
%$x$ is ``not too far'' from $a$.
%If one ``zooms in'' on $f$ at $x=a$ sufficiently, then $f$ and the linear
%approximation are nearly indistinguishable. As a first example, we
%will see how linear approximations allow us to make approximate
%``difficult'' computations.
%\end{example}
%
%
%
%%%%%%% Aproximation of \sqrt[3]{50} number 1
%\begin{example}
%Use a linear approximation of $f(x) =\sqrt[3]{x}$ at $x=64$ to
%approximate $\sqrt[3]{50}$.
%\begin{explanation}
%To start, write
%\[
%\ddx f(x) = \ddx x^{1/3} = \frac{1}{\answer[given]{3x^{2/3}}}.
%\]
%So our linear approximation is
%\begin{align*}
%\l(x) &= \frac{1}{3\cdot 64^{2/3}} (x-64) + \answer[given]{4} \\
%&= \frac{1}{\answer[given]{48}} (x-64) + 4\\
%&= \answer[given]{1/48}x + \answer[given]{8/3}.
%\end{align*}
%
%\begin{question}
%Using the linear approximation just found (and the Desmos calculator below), 
%
%\[
%\sqrt[3]{50} \approx \answer[tolerance=0.000001]{3.70833333}.
%\]
%\begin{hint}
%Remember to round to 6 decimal places.
%\end{hint}
%\end{question}
%
%\[
%\graph[panel]{}
%\]
%
%
%
%
%
%\begin{image}
%\begin{tikzpicture}
%	\begin{axis}[
%            xmin=1,xmax=100,ymin=0,ymax=5,
%            axis lines=center,
%            xlabel=$x$, ylabel=$y$,
%            every axis y label/.style={at=(current axis.above origin),anchor=south},
%            every axis x label/.style={at=(current axis.right of origin),anchor=west},
%          ]        
%          \addplot [very thick, penColor, samples=150,smooth,domain=(0:100)] {x^(1/3))};
%          \addplot [very thick, penColor2, domain=(0:100)] {x/48+8/3};
%          \addplot [textColor,dashed] plot coordinates {(64,0) (64,4)};
%          \addplot [textColor,dashed] plot coordinates {(0,4) (64,4)};
%          \node at (axis cs:20,2.3) [penColor] {$f$};
%          \node at (axis cs:20,3.3) [penColor2] {$\l$};
%          \addplot[color=penColor3,fill=penColor3,only marks,mark=*] coordinates{(64,4)};  %% closed hole            
%        \end{axis}
%\end{tikzpicture}
%\end{image}
%
%Now let's use our approximation and true value to find our approximation's error.
%
%
%Use the Desmos widget below to determine the error of your approximation for $\sqrt[3]{50}$, noting that $\sqrt[3]{50} = f(50)$.
%
%\[
%\graph[panel]{f(x) = x^{1/3}, l(x) = (1/48) x + (8/3)}
%\]
%
%\begin{question}
%The approximation is \wordChoice{\choice[correct]{above}\choice{below}} the true 
%answer by $\answer[tolerance=0.000001]{0.024302}$ units.
%
%\begin{hint}
%Note that $f(50)$ when plugged into Desmos will give $
%sqrt[3]{50}$.
%\end{hint}
%\begin{hint}
%What will be $\l(50)$ give you in the Desmos calculator?
%\end{hint}
%\begin{hint}
%How can you use the Desmos calculator to find the difference of these two numbers?
%\end{hint}
%\begin{hint}
%What does $f(50)-\l(50)$ give you? Is that what you want?
%\end{hint}
%\end{question}
%\end{explanation}
%\end{example}
%
%
%With modern calculators and computing software it may not appear
%necessary to use linear approximations. In fact they are quite
%useful. In cases requiring an explicit numerical approximation, they
%allow us to get a quick rough estimate which can be used as a
%``reality check'' on a more complex calculation. In some complex
%calculations involving functions, the linear approximation makes an
%otherwise intractable calculation possible, without serious loss of
%accuracy.
%
%
%
%
%%% Approximation using cube root of 27 instead.
%\begin{example}
%Now use a linear approximation of $f(x) =\sqrt[3]{x}$ at $\underline{x=27}$ to
%approximate $\sqrt[3]{50}$.
%\begin{explanation}
%To start, write
%\[
%\ddx f(x) = \ddx x^{1/3} = \frac{1}{\answer[given]{3x^{2/3}}}.
%\]
%By using our formula for linear approximations, our linear 
%approximation for $f(x)=\sqrt[3]{x}$ at $x=27$ is
%\begin{align*}
%m(x) &= \answer[given]{1/27}x + \answer[given]{2}.
%\end{align*}
%
%\[
%\graph[panel]{}
%\]
%
%\begin{question}
%Using the linear approximation just found (and the Desmos calculator above),
%\[
%\sqrt[3]{50} \approx \answer[tolerance=0.0000005]{3.851852}.
%\]
%\begin{hint}
%Remember to round to 6 decimal places.
%\end{hint}
%\end{question}
%
%\begin{image}
%\begin{tikzpicture}
%	\begin{axis}[
%            xmin=1,xmax=100,ymin=0,ymax=5,
%            axis lines=center,
%            xlabel=$x$, ylabel=$y$,
%            every axis y label/.style={at=(current axis.above origin),anchor=south},
%            every axis x label/.style={at=(current axis.right of origin),anchor=west},
%          ]        
%          \addplot [very thick, penColor, samples=150,smooth,domain=(0:100)] {x^(1/3))};
%          \addplot [very thick, penColor2, domain=(0:100)] {x/27+2};
%          \addplot [textColor,dashed] plot coordinates {(27,0) (27,3)};
%          \addplot [textColor,dashed] plot coordinates {(0,3) (27,3)};
%          \node at (axis cs:50,3.3) [penColor] {$f$};
%          \node at (axis cs:50,4.3) [penColor2] {$m$};
%          \addplot[color=penColor3,fill=penColor3,only marks,mark=*] coordinates{(27,3)};  %% closed hole            
%        \end{axis}
%\end{tikzpicture}
%\end{image}
%
%Now let's use our approximation and true value to find our approximation's error.
%
%Use the Desmos widget below to determine the error of your approximation for $\sqrt[3]{50}$, noting that $\sqrt[3]{50} = f(50)$.
%
%\[
%\graph[panel]{f(x) = x^{1/3}, m(x) = (1/27) x + 2}
%\]
%
%\begin{question}
%The approximation is \wordChoice{\choice[correct]{above}\choice{below}} the true 
%answer by $\answer[tolerance=0.000001]{0.167820}$ units.
%
%\begin{hint}
%Note that $f(50)$ when plugged into Desmos will give $
%sqrt[3]{50}$.
%\end{hint}
%\begin{hint}
%What will be $m(50)$ give you in the Desmos calculator?
%\end{hint}
%\begin{hint}
%How can you use the Desmos calculator to find the difference of these two numbers?
%\end{hint}
%\begin{hint}
%What does $f(50)-m(50)$ give you? Is that what you want?
%\end{hint}
%\end{question}
%
%
%\end{explanation}
%\end{example}
%
%Notice that we approximated $\sqrt[3]{50}$ using two different linear approximations of $\sqrt[3]{x}$ and hence got two different approximations of $\sqrt[3]{50}$.
%
%
%%%% Comparison of approximations
%\begin{question}
%Compare your linear approximations when using $x=64$ and $x=27$. Which linear approximation of $\sqrt[3]{x}$ gives a better approximation of $\sqrt[3]{50}$?
%
%\begin{multipleChoice}
%   \choice{The approximation using $x=27$, because $27$ is closer to $50$ than $64$.}
%   \choice{The approximation using $x=27$, because $64$ is closer to $50$ than $27$.}
%   \choice[correct]{The approximation using $x=64$, because $64$ is closer to $50$ than $27$.}
%   \choice{The approximation using $x=64$, because $27$ is closer to $50$ than $64$.}
%\end{multipleChoice}
%
%\begin{image}
%\begin{tikzpicture}
%	\begin{axis}[
%            xmin=20,xmax=70,ymin=2.5,ymax=4.5,
%            axis lines=center,
%            xlabel=$x$, ylabel=$y$,
%            every axis y label/.style={at=(current axis.above origin),anchor=south},
%            every axis x label/.style={at=(current axis.right of origin),anchor=west},
%          ]        
%          \addplot [very thick, penColor, samples=250,smooth,domain=(20:70)] {x^(1/3))};
%          \addplot [very thick, penColor2, domain=(20:70)] {x/27+2};
%          \addplot [very thick, penColor4, domain=(20:70)] {x/48+8/3};
%          \addplot [textColor,dashed] plot coordinates {(50,0) (50,3.68403149864)};         
%          \node at (axis cs:42,3.1) [penColor] {$f$};
%          \node at (axis cs:55,4.2) [penColor2] {$m$};
%          \node at (axis cs:35,3.5) [penColor4] {$\l$};
%          \addplot[color=penColor3,fill=penColor3,only marks,mark=*] coordinates{(50,3.68403149864)};  %% closed hole            
%        \end{axis}
%\end{tikzpicture}
%\end{image}
%
%As a warning: not all approximations are over estimates! This just came about
%from this example. Some linear approximations are over-estimates. Some are
%under-estimates. Some vary for different points you are approximating at!
%\end{question}
%
%
%
%%% Approximation for sine function
%\begin{example}
%Use a linear approximation of $f(x) =\sin(x)$ at $x=0$ to approximate
%$\sin(0.3)$.
%\begin{explanation}
%To start, write
%\[
%\ddx f(x) = \answer[given]{\cos(x)},
%\]
%so our linear approximation is
%\begin{align*}
%\l(x) &= \answer[given]{\cos(0)}\cdot(x-\answer[given]{0}) + \answer[given]{0}\\
%&= \answer[given]{x}.
%\end{align*}
%\begin{image}
%%\begin{marginfigure}
%\begin{tikzpicture}
%	\begin{axis}[
%            xmin=-1.6,xmax=1.6,ymin=-1.5,ymax=1.5,
%            axis lines=center,
%            xtick={-1.57, 0, 1.57},
%            xticklabels={$-\pi/2$, $0$, $\pi/2$},
%            ytick={-1,1},
%            unit vector ratio*=1 1 1,
%            xlabel=$x$, ylabel=$y$,
%            every axis y label/.style={at=(current axis.above origin),anchor=south},
%            every axis x label/.style={at=(current axis.right of origin),anchor=west},
%          ]        
%          \addplot [very thick, penColor, samples=100,smooth, domain=(-1.6:1.6)] {sin(deg(x))};
%          \addplot [very thick, penColor2, samples=100,smooth] {x};
%          \node at (axis cs:1,.6) [penColor] {$f$};
%          \node at (axis cs:-1,-1.2) [penColor2] {$\l$};
%          \addplot[color=penColor3,fill=penColor3,only marks,mark=*] coordinates{(0,0)};  %% closed hole          
%        \end{axis}
%\end{tikzpicture}
%%\caption{A linear approximation of $f(x) = \sin(x)$ at $x=0$.}
%%\label{figure:la sin}
%%\end{marginfigure}
%\end{image}
%Hence the linear approximation for $\sin(x)$ at $x=0$ is $\l(x) = x$,
%and so $\l(0.3) = 0.3$.  Comparing this to $\sin(.3) \approx 0.295$,
%we see that the approximation is quite good. For this reason, it is common
%to approximate $\sin(x)$ with its linear approximation $\l(x) = x$
%when $x$ is near zero.  
%%see Figure~\ref{figure:la sin}.
%\end{explanation}
%\end{example}
%
%Let's apply this notion of linear approximations in a physics application to
%answer a real-life question!







%%% Approximating periodicity of a pendulum!
%%% inspired from http://www.acs.psu.edu/drussell/Demos/Pendulum/Pendula.html
\begin{example}

Consider a pendulum with ball $B$ with mass $m$ hanging from a string
with length $L$ hanging from a block of woods at pivot point $P$.
Below is a picture of the pendulum that is displaced with angle
$\theta$ from equilibrium.

%% Picture of pendulum
%Edited from http://tex.stackexchange.com/questions/219038/tikz-draw-angle-with-label-between-lines
\begin{center}
\usetikzlibrary{calc,patterns,angles,quotes}
\begin{tikzpicture}
    \coordinate (origo) at (0,0);
    \coordinate (pivot) at (1,5);

    % draw axes
    \fill[black] (origo) circle (0.05);
    \node[below left=1pt of {(0,0)}] {Point $P$};
    \draw[thick,gray,->] (origo) -- ++(4,0) node[black,right] {};
    \draw[thick,dashed, gray,->] (origo) -- ++(0,-4) node (mary) [black,below] {};

    % draw roof
    \fill[pattern = north east lines] ($ (origo) + (-1,0) $) rectangle ($ (origo) + (1,0.5) $);
    \draw[thick] ($ (origo) + (-1,0) $) -- ($ (origo) + (1,0) $);

    \draw[thick] (origo) -- ++(300:3) coordinate (bob);
    \fill (bob) circle (0.2);
    \node[below left=5pt of {(1.5,-2.5)}] {$B$};

    \pic [draw, ->, "$\theta$", angle eccentricity=1.5] {angle = mary--origo--bob};
  \end{tikzpicture}
\end{center}

In this model when the mass is directly below the pivot point $P$
(on the dashed line), the string is in equilibrium and doesn't move.

Suppose we'd like to find out how long it takes this pendulum
to swing back and forth (one period).

\begin{explanation}
First of all to find the period of the pendulum, we need to understand
how the pendulum moves. Note that the position of the pendulum is completely determined
by the angle of displacement $\theta$ (Assuming that the string is always tight).
If we move the ball of the pendulum to a certain $\theta\neq 0$ and let go,
the pendulum will swing back and forth. Therefore the angle $\theta$
is a function of time, call it $t$. So how does this bad-boy move?

Newton's second law of motion implies the following 
relationship holds in this system:

\[
-m g L \phantom{.} \sin(\theta) = m L^2 \phantom{.}\dfrac{d^2 \theta}{d t^2}
\]
where $g$ is earth's gravitational constant.

Our goal is to solve this differential equation to find what the function
$\theta$ is. Once we find $\theta$, we can find the period of the pendulum's swing.

Let's get to work. Manipulate the above equation so that `$0$'
is on one side and the $\dfrac{d^2 \theta}{d t^2}$ term is positive.

\[
0 = \answer[given]{m L^2} \dfrac{d^2 \theta}{d t^2}  + \answer[given]{m g L \sin(\theta)}
\]

Note that $L,m,$ and $g$ are constant and greater than $0$, so we can
divide the equality by any of these terms to get an equivalent statement.
Manipulate the above equality so that the coefficient of $\dfrac{d^2 \theta}{d t^2}$
is $1$. Be sure to simplify fractions.

\[
0 = \dfrac{d^2 \theta}{d t^2}  + \dfrac{\answer[given]{g}}{\answer[given]{L}} \phantom{.} \sin(\theta)
\]

\begin{problem}
Does the above relationship depend on the length of the string?
\begin{multipleChoice}
\choice[correct]{Yes}
\choice{No}
\end{multipleChoice}
\begin{feedback}
There's no way to eliminate $L$ from the equation, so the solution
to the differential equation depends on the length of the string $L$.
\end{feedback}
\end{problem}

\begin{problem}
Does the above relationship depend on the gravitational constant?
\begin{multipleChoice}
\choice[correct]{Yes}
\choice{No}
\end{multipleChoice}
\begin{feedback}
There's no way to eliminate $g$ from the equation, so the solution
to the differential equation depends on earth's gravitational
constant $g$.
\end{feedback}
\end{problem}

\begin{problem}
Does the above relationship depend on the mass of ball?
\begin{multipleChoice}
\choice{Yes}
\choice[correct]{No}
\end{multipleChoice}
\begin{feedback}
By dividing by $m$, we were able to eliminate the mass $m$ from
the equation. Therefore the solution to the differential equation
(the position/path of the ball) doesn't depend
on the mass of the ball. Even though $m$ was in the initial equality,
it isn't relevant to the solution. Holy smokes!
\end{feedback}
\end{problem}

Now let's try to solve this differential equation!

\[
0 = \dfrac{d^2 \theta}{d t^2}  + \dfrac{g}{L} \sin(\theta)
\]

We will utilize the fact that $\theta$, the angle of displacement of the ball $B$),
is a function of time, $t$. We are trying to solve the differential
equation above so that we can find the period of the swing of the pendulum.

Manipulate the above equation to so that $\dfrac{d^2 \theta}{d t^2}$ is in terms of $\theta$.

\[
\dfrac{d^2 \theta}{d t^2}  = \answer[given]{-g\sin(\theta)/L}
\]

The differential equation is too tough for us to solve at the moment,
but we can solve something similar.


Let's solve this differential equation for small displacements
of $\theta$. We will replace `$\sin(\theta)$' with a linear approximation
of $\sin(\theta)$.


\begin{question}
If we want to solve this differential displacements from equilibrium, at value of $\theta$ should we approximate $\sin(\theta)$?


We find a linear approximation of $\sin(\theta)$ at $\theta = \answer[given]{0}$.

\end{question}

\begin{question}
What is the linear approximation of $\sin(\theta)$ at the above $\theta$ value?

\[
\sin(\theta) \approx \answer[given]{\theta}
\]

\begin{hint}
Remember $\theta$ is the variable, not $x$.
\end{hint}
\begin{hint}
You've already found this early this earlier in this lab.
\end{hint}
\end{question}

\begin{remark}
The above approximation is actually a surprisingly good for
$|\theta|\leq \frac{\pi}{12} = 15^\circ$, so in this context, "small" angles
are angles that deviate no more that $15^\circ$ from equilibrium, FYI.
\end{remark}

Therefore we are trying to find a solution to

\[
\dfrac{d^2 \theta}{d t^2}  = -\dfrac{g}{L} \theta
\]

or for the constant $C=\frac{g}{L}$,

\[
\dfrac{d^2 \theta}{d t^2}  = -C \theta.
\]

Another way to phrase this is that $\theta$ is a functions whose
second derivative is $-1$ times a scalar $C$ of the original function.


\textbf{Question.}  Do you know any types of functions whose \emph{second} derivative
is $-1$ times itself (disregarding the $C$)? Which of the following have at least one function that has this property.

\begin{selectAll}
\choice[correct]{Polynomials of degree $0$ (Constant functions)}
\choice{Polynomials of degree $1$ or bigger}
\choice[correct]{Trigonometric functions}
\end{selectAll}


\begin{problem}
Which polynomial $f(t)$ has the property that it's second
derivative is $-1$ times itself i.e. if
$\theta=f(t)$, then $\dfrac{d^2 \theta}{d t^2} = f''(t) = -f(t)$?


The polynomial function $f(t) = \answer[given]{0}$ has the property that $f''(t) = -f(t)$.

\begin{hint}
Don't over think it.
\end{hint}
\begin{hint}
It is polynomial in the variable $t$.
\end{hint}
\begin{hint}
It is actually a constant function.
\end{hint}
\end{problem}

This is our equilibrium solution! This function has $\theta=0$ at $t=0$. In other
words, this is our solution when we put the ball directly below the point
that the string is hanging from. This is no surprise though since this
is the equilibrium. What we \emph{really} want is to find out what happens
when we place the ball somewhere where the ball actually swings.


\textbf{Question.}  Which of the following trigonometric functions have this property
($\dfrac{d^2 \theta}{d t^2} = -\theta$)? Select all such functions.
\begin{selectAll}
\choice[correct]{$\sin(t)$}
\choice[correct]{$\cos(t)$}
\choice{$\tan(t)$}
\choice{$\csc(t)$}
\choice{$\sec(t)$}
\choice{$\cot(t)$}
\end{selectAll}


\begin{question}
Since we are simulating the motion of the pendulum when letting go of the pendulum from
a stand still, which of the functions you picked above would accurately simulate this situation?
\begin{selectAll}
\choice{$\sin(t)$}
\choice[correct]{$\cos(t)$}
\choice{$\tan(t)$}
\choice{$\csc(t)$}
\choice{$\csc(t)$}
\choice{$\csc(t)$}
\end{selectAll}
\begin{hint}
We want that the pendulum is not moving at $t=0$.
\end{hint}
\begin{hint}
We want a function $f(t)$ such that $f'(t) = 0$ at $t=0$.
\end{hint}
\end{question}

We're almost there! We know what the function generally looks like, but we
need to check the fine details to make sure we answer the question we set out to answer.

We want to solve the differential equation

\[
\dfrac{d^2 \theta}{d t^2}  = -\dfrac{g}{L} \theta
\]

or for the constant $C=\frac{g}{L}$,

\[
\dfrac{d^2 \theta}{d t^2}  = -C\theta.
\]

For $\theta = \cos(ax+b)$

\begin{align*}
    \dfrac{d^2}{d t^2}(\theta) &= \dfrac{d^2}{d t^2}\Big(\cos(a x + b)\Big)\\
    &= \dfrac{d}{d t} \Big( \answer[given]{- a \sin(a x+b)}\Big)\\
    &= \answer[given]{-a^2 \cos(a x+b)}\\
    &= -a^2 \theta 
\end{align*}

\begin{question}
So if we are looking for a solution to $\dfrac{d^2 \theta}{d t^2}  = -\dfrac{g}{L} \theta$,
then solutions to this differential equation is $\theta = \cos(ax+b)$ where

\[
b = \answer[given]{0}
\]

and 

\[
a^2 = \answer[given]{g/L}.
\]

Assuming $a>0$, then 

\[
a = \dfrac{\answer[given]{\sqrt{g}}}{\answer[given]{\sqrt{L}}}
\]
\end{question}

\begin{question}
Given your solution to the differential equation above, what is the period of a ball with mass $m$ on a string with length $L$ which is displaced at "small" angles?

\[
(\answer[given]{2\pi})\sqrt{\Big(\answer[given]{L/g}\Big)}
\]
\begin{hint}
For $A\sin(ax+b)$ or $A\cos(ax+b)$, the period of the function is $\dfrac{2\pi}{a}$.
\end{hint}
\end{question}

If the ball starts at an initial angle of displacement $\theta_0$
(where $|\theta_0|<\frac{\pi}{12}= 15^\circ$), we can describe
the position of the ball at any given time by

\[
\theta(t) = \theta_0 \cos\Big(\sqrt{\frac{g}{L}} t\Big).
\]
Pretty neat, huh?

\begin{question}
Given a ball with mass $0.6$ kilograms hanging from a string with length $7$ meters, what is the period of a pendulum swing when the swing doesn't deviate more than $\frac{\pi}{12}$ radians from a free hang (where gravitational constant $9.8$ meters per second)?


The period is approximately $\answer[tolerance=0.01]{2*3.1415*0.84515}$.
\end{question}
\end{explanation}
\end{example}



\end{document}