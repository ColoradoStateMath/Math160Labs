\documentclass[handout,nooutcomes]{ximera}
%handout
%wordchoicegiven
%space
%nooutcomes
\title{Math 160 Lab 2}
\author{Ben Sencindiver} %Used Bart Snapp and Jim Fowler's mooculus textbook as a guide
%\usepackage{todonotes}

\newcommand{\todo}{}

\usepackage{tkz-euclide}
\tikzset{>=stealth} %% cool arrow head
\tikzset{shorten <>/.style={ shorten >=#1, shorten <=#1 } } %% allows shorter vectors

\usetikzlibrary{backgrounds} %% for boxes around graphs
\usetikzlibrary{shapes,positioning}  %% Clouds and stars
\usetikzlibrary{matrix} %% for matrix
\usetkzobj{all}
\usepackage[makeroom]{cancel} %% for strike outs
%\usepackage{mathtools} %% for pretty underbrace % Breaks Ximera
\usepackage{multicol}


\newcommand{\RR}{\mathbb R}
%\renewcommand{\d}{\,d\!}
\renewcommand{\d}{\mathop{}\!d\!}
\newcommand{\dd}[2][]{\frac{\d #1}{\d #2}}
\renewcommand{\l}{\ell}
\newcommand{\ddx}{\frac{d}{\d x}}

\newcommand{\zeroOverZero}{\ensuremath{\boldsymbol{\tfrac{0}{0}}}}
\newcommand{\inftyOverInfty}{\ensuremath{\boldsymbol{\tfrac{\infty}{\infty}}}}
\newcommand{\zeroOverInfty}{\ensuremath{\boldsymbol{\tfrac{0}{\infty}}}}
\newcommand{\zeroTimesInfty}{\ensuremath{\small\boldsymbol{0\cdot \infty}}}
\newcommand{\inftyMinusInfty}{\ensuremath{\small\boldsymbol{\infty - \infty}}}
\newcommand{\oneToInfty}{\ensuremath{\boldsymbol{1^\infty}}}
\newcommand{\zeroToZero}{\ensuremath{\boldsymbol{0^0}}}
\newcommand{\inftyToZero}{\ensuremath{\boldsymbol{\infty^0}}}


\newcommand{\numOverZero}{\ensuremath{\boldsymbol{\tfrac{\#}{0}}}}
\newcommand{\dfn}{\textbf}
%\newcommand{\unit}{\,\mathrm}
\newcommand{\unit}{\mathop{}\!\mathrm}
\newcommand{\eval}[1]{\bigg[ #1 \bigg]}
\renewcommand{\epsilon}{\varepsilon}
\renewcommand{\iff}{\Leftrightarrow}

\DeclareMathOperator{\arccot}{arccot}
\DeclareMathOperator{\arcsec}{arcsec}
\DeclareMathOperator{\arccsc}{arccsc}
\DeclareMathOperator{\si}{Si}


\colorlet{textColor}{black} 
\colorlet{background}{white}
\colorlet{penColor}{blue!50!black} % Color of a curve in a plot
\colorlet{penColor2}{red!50!black}% Color of a curve in a plot
\colorlet{penColor3}{red!50!blue} % Color of a curve in a plot
\colorlet{penColor4}{green!50!black} % Color of a curve in a plot
\colorlet{penColor5}{orange!80!black} % Color of a curve in a plot
\colorlet{fill1}{penColor!20} % Color of fill in a plot
\colorlet{fill2}{penColor2!20} % Color of fill in a plot
\colorlet{fillp}{fill1} % Color of positive area
\colorlet{filln}{penColor2!20} % Color of negative area
\colorlet{fill3}{penColor3!20} % Fill
\colorlet{fill4}{penColor4!20} % Fill
\colorlet{fill5}{penColor5!20} % Fill
\colorlet{gridColor}{gray!50} % Color of grid in a plot

\pgfmathdeclarefunction{gauss}{2}{% gives gaussian
  \pgfmathparse{1/(#2*sqrt(2*pi))*exp(-((x-#1)^2)/(2*#2^2))}%
}



\newcommand{\fullwidth}{}
\newcommand{\normalwidth}{}


%% makes a snazzy t-chart for evaluating functions
%\newenvironment{tchart}{\rowcolors{2}{}{background!90!textColor}\array}{\endarray}

%%This is to help with formatting on future title pages.
\newenvironment{sectionOutcomes}{}{} 


\outcome{Define linear approximation as an application of the tangent to a curve.}
\outcome{Find the linear approximation to a function at a point and use it to approximate the function value.}
\outcome{Identify when a linear approximation can be used.}
\outcome{Label a graph with the appropriate quantities used in linear approximation.}
\outcome{Find the error of a linear approximation.}

\begin{document}

\section{Math 160 Lab 2 \\ Linear Approximation}

%% Have to edit the date here.
\begin{abstract}
This is Lab 2 for Math 160 - Due Wednesday, March 8, 2017 at 5:00PM MST.
This lab will cover ``linear approximations'': what are they,
how do you compute them, and how are they useful.\\

Throughout this lab unless stated otherwise, compute all values to $6$ decimal places.
\end{abstract}

\maketitle


A linear approximation of $f$ will be a ``good'' approximation for values of
$x$ that are ``not too far'' from $a$.
If one mentally ``zooms in'' on $f$ at $x=a$ sufficiently, then $f$ and the linear
approximation are nearly indistinguishable. As a first example, we
will see how linear approximations allow us to approximate
``difficult'' computations.


%%%%%% Aproximation of \sqrt[3]{50} number 1
\begin{example}
Use a linear approximation of $f(x) =\sqrt[3]{x}$ at $x=64$ to
approximate $\sqrt[3]{50}$.\\

\begin{explanation}
To start, write
\[
\ddx f(x) = \ddx x^{1/3} = \frac{1}{\answer[given]{3x^{2/3}}}.
\]
So our linear approximation is
\begin{align*}
\l(x) &= \frac{1}{3\cdot 64^{2/3}} (x-64) + \answer[given]{4} \\
&= \frac{1}{\answer[given]{48}} (x-64) + 4\\
&= \answer[given]{1/48}x + \answer[given]{8/3}.
\end{align*}



\textbf{Question. } Using the linear approximation just found (and
the Desmos calculator below),

\[
\sqrt[3]{50} \approx \answer[tolerance=0.000001]{3.70833333}.
\]

\smallskip
\textbf{Hint}: Though you could do this computation by hand, you can use
the desmos calculator below to compute the number for you! You may want
to change $m$ and $b$ so that $l(x)$ is the linear approximation you
found above (i.e. $\l(x)$). Remember to round to 6 decimal places.
(Click the blue arrows on the right to reveal more hints)

\begin{expandable}
Looking at your approximation above, what should $m$ and $b$ be?
\end{expandable}
\begin{expandable}
Once you pick values for m and b, you can evaluate your function
at x=12 (for example) by typing `l(12)' into one of the blank boxes below l(x).
You must have l(x) defined on a different line. The graph of you line
should be visible on the graph in order to do this.
\end{expandable}
\begin{expandable}
If l(x) is a linear approximation of a function and you want to approximate
the function at 14, type `l(14)' into one of the boxes that doesn't define what l(x) is.
\end{expandable}

\[
\graph[panel]{l(x)=mx+b}
\]
\hrule

\medskip

To get a better idea what this linear approximation looks like relative to
$f(x)=\sqrt{x}$ and how the true value differs form our approximations.
This graph tells us whether the approximations will be over estimates or
under estimates.\\



\begin{image}
\begin{tikzpicture}
	\begin{axis}[
            xmin=1,xmax=100,ymin=0,ymax=5,
            axis lines=center,
            xlabel=$x$, ylabel=$y$,
            every axis y label/.style={at=(current axis.above origin),anchor=south},
            every axis x label/.style={at=(current axis.right of origin),anchor=west},
          ]        
          \addplot [very thick, penColor, samples=150,smooth,domain=(0:100)] {x^(1/3))};
          \addplot [very thick, penColor2, domain=(0:100)] {x/48+8/3};
          \addplot [textColor,dashed] plot coordinates {(64,0) (64,4)};
          \addplot [textColor,dashed] plot coordinates {(0,4) (64,4)};
          \node at (axis cs:20,2.3) [penColor] {$f$};
          \node at (axis cs:20,3.3) [penColor2] {$\l$};
          \addplot[color=penColor3,fill=penColor3,only marks,mark=*] coordinates{(64,4)};  %% closed hole            
        \end{axis}
\end{tikzpicture}
\end{image}

Now let's use our approximation and true value of $\sqrt[3]{50}$ to
find our approximation's error.\\


Use the Desmos widget below to determine the error of your approximation for $\sqrt[3]{50}$, noting that $\sqrt[3]{50} = f(50)$.

\textbf{Question. } The approximation is \wordChoice{\choice[correct]{larger}\choice{smaller}}
than the true value of $\sqrt[3]{50}$ by $\answer[tolerance=0.000001]{0.024302}$ units.

\textbf{Hint}: Note that $f(50)$ when plugged into Desmos will give $\sqrt[3]{50}$.
(Click the blue arrows on the right to reveal more hints.)

\begin{expandable}
What will be `l(50)' give you in the Desmos calculator?
\end{expandable}
\begin{expandable}
How can you use the Desmos calculator to find the difference of these two numbers?
\end{expandable}
\begin{expandable}
What does `f(50)-l(50)' give you? Is that what you want?
\end{expandable}
\begin{expandable}
Make sure the value you are typing in is positive.
\end{expandable}


\[
\graph[panel]{f(x) = x^{1/3}, l(x) = (1/48) x + (8/3)}
\]


\end{explanation}
\end{example}

With modern calculators and computing software, it may not appear
necessary to use linear approximations. In fact they are quite
useful. In cases requiring an explicit numerical approximation, they
allow us to get a quick rough estimate which can be used as a
``reality check'' on a more complex calculation. In some complex
calculations involving functions, the linear approximation makes an
otherwise intractable calculation possible, without serious loss of
accuracy.



%% Approximation using cube root of 27 instead.
\begin{example}
Now use a linear approximation of $f(x) =\sqrt[3]{x}$ at $\underline{x=27}$ to
approximate $\sqrt[3]{50}$.
\begin{explanation}
To start, write
\[
\ddx f(x) = \ddx x^{1/3} = \frac{1}{\answer[given]{3x^{2/3}}}.
\]
By using our formula for linear approximations, our linear 
approximation for $f(x)=\sqrt[3]{x}$ at $x=27$ is
\begin{align*}
m(x) &= \answer[given]{1/27}x + \answer[given]{2}.
\end{align*}


\textbf{Question. } Using the linear approximation just found (and the Desmos calculator below),
\[
\sqrt[3]{50} \approx \answer[tolerance=0.0000005]{3.851852}.
\]
\smallskip
\textbf{Hint}: Though you could do this computation by hand, you can use
the desmos calculator below to compute the number for you! You may want
to change $m$ and $b$ so that $l(x)$ is the linear approximation you
found above (i.e. $\l(x)$). Remember to round to 6 decimal places.
(Click the blue arrows on the right to reveal more hints)

\begin{expandable}
Looking at your approximation above, what should $m$ and $b$ be?
\end{expandable}
\begin{expandable}
Once you pick values for m and b, you can evaluate your function
at x=12 (for example) by typing `l(12)' into one of the blank boxes below l(x).
You must have l(x) defined on a different line. The graph of you line
should be visible on the graph in order to do this.
\end{expandable}
\begin{expandable}
If l(x) is a linear approximation of a function and you want to approximate
the function at 14, type `l(14)' into one of the boxes that doesn't define what l(x) is.
\end{expandable}
\[
\graph[panel]{l(x) = mx+b}
\]

\medskip

Now let's use our approximation and true value to find our approximation's error.

Use the Desmos widget below to determine the error of your approximation for $\sqrt[3]{50}$, noting that $\sqrt[3]{50} = f(50)$.

\textbf{Hint}: Note that $f(50)$ when plugged into Desmos will give $\sqrt[3]{50}$.
(Click the blue arrows on the right to reveal more hints.)

\begin{expandable}
What will be `l(50)' give you in the Desmos calculator?
\end{expandable}
\begin{expandable}
How can you use the Desmos calculator to find the difference of these two numbers?
\end{expandable}
\begin{expandable}
What does `f(50)-l(50)' give you? Is that what you want?
\end{expandable}
\begin{expandable}
Make sure the value you are typing in is positive.
\end{expandable}

%%%%%%%%%%%%%%%%%%%%%%%%%%%%%%%%%%%%%%%%%%

\textbf{Question. } The approximation is \wordChoice{\choice[correct]{above}\choice{below}} the true 
answer by $\answer[tolerance=0.000001]{0.167820}$ units.

\begin{expandable}
Hint: Note that $f(50)$ when plugged into Desmos will give $sqrt[3]{50}$.
\begin{expandable}
What will be $m(50)$ give you in the Desmos calculator?
\begin{expandable}
How can you use the Desmos calculator to find the difference of these two numbers?
\begin{expandable}
What does $f(50)-m(50)$ give you? Is that what you want?
\end{expandable}
\end{expandable}
\end{expandable}
\end{expandable}

\[
\graph[panel]{f(x) = x^{1/3}, m(x) = (1/27) x + 2}
\]


\end{explanation}
\end{example}

Notice that we approximated $\sqrt[3]{50}$ using two different
linear approximations of $\sqrt[3]{x}$ and hence got two different
approximations of $\sqrt[3]{50}$.

%%% Comparison of approximations
\begin{question}
Compare your linear approximations when using $x=64$ and $x=27$. Which linear approximation of $\sqrt[3]{x}$ gives a better approximation of $\sqrt[3]{50}$?

\begin{multipleChoice}
   \choice{The approximation using $x=27$, because $27$ is closer to $50$ than $64$.}
   \choice{The approximation using $x=27$, because $64$ is closer to $50$ than $27$.}
   \choice[correct]{The approximation using $x=64$, because $64$ is closer to $50$ than $27$.}
   \choice{The approximation using $x=64$, because $27$ is closer to $50$ than $64$.}
\end{multipleChoice}

\begin{image}
\begin{tikzpicture}
	\begin{axis}[
            xmin=20,xmax=70,ymin=2.5,ymax=4.5,
            axis lines=center,
            xlabel=$x$, ylabel=$y$,
            every axis y label/.style={at=(current axis.above origin),anchor=south},
            every axis x label/.style={at=(current axis.right of origin),anchor=west},
          ]        
          \addplot [very thick, penColor, samples=250,smooth,domain=(20:70)] {x^(1/3))};
          \addplot [very thick, penColor2, domain=(20:70)] {x/27+2};
          \addplot [very thick, penColor4, domain=(20:70)] {x/48+8/3};
          \addplot [textColor,dashed] plot coordinates {(50,0) (50,3.68403149864)};         
          \node at (axis cs:42,3.1) [penColor] {$f$};
          \node at (axis cs:55,4.2) [penColor2] {$m$};
          \node at (axis cs:35,3.5) [penColor4] {$\l$};
          \addplot[color=penColor3,fill=penColor3,only marks,mark=*] coordinates{(50,3.68403149864)};  %% closed hole            
        \end{axis}
\end{tikzpicture}
\end{image}

As a warning: not all approximations are over estimates! This just came about
from this example. Some linear approximations are over-estimates. Some are
under-estimates. Some vary for different points you are approximating at!
\end{question}





%% Approximation for sine function
\begin{example}
Use a linear approximation of $f(x) =\sin(x)$ at $x=0$ to approximate
$\sin(0.3)$.
\begin{explanation}
To start, write
\[
\ddx f(x) = \answer[given]{\cos(x)},
\]
so our linear approximation is
\begin{align*}
\l(x) &= \answer[given]{\cos(0)}\cdot(x-\answer[given]{0}) \\
+ \answer[given]{0}\\
&= \answer[given]{x}.
\end{align*}
\begin{image}
%\begin{marginfigure}
\begin{tikzpicture}
	\begin{axis}[
            xmin=-1.6,xmax=1.6,ymin=-1.5,ymax=1.5,
            axis lines=center,
            xtick={-1.57, 0, 1.57},
            xticklabels={$-\pi/2$, $0$, $\pi/2$},
            ytick={-1,1},
            unit vector ratio*=1 1 1,
            xlabel=$x$, ylabel=$y$,
            every axis y label/.style={at=(current axis.above origin),anchor=south},
            every axis x label/.style={at=(current axis.right of origin),anchor=west},
          ]        
          \addplot [very thick, penColor, samples=100,smooth, domain=(-1.6:1.6)] {sin(deg(x))};
          \addplot [very thick, penColor2, samples=100,smooth] {x};
          \node at (axis cs:1,.6) [penColor] {$f$};
          \node at (axis cs:-1,-1.2) [penColor2] {$\l$};
          \addplot[color=penColor3,fill=penColor3,only marks,mark=*] coordinates{(0,0)};  %% closed hole          
        \end{axis}
\end{tikzpicture}
%\caption{A linear approximation of $f(x) = \sin(x)$ at $x=0$.}
%\label{figure:la sin}
%\end{marginfigure}
\end{image}
Hence the linear approximation for $\sin(x)$ at $x=0$ is $\l(x) = x$,
and so $\l(0.3) = 0.3$.  Comparing this to $\sin(.3) \approx 0.295$,
we see that the approximation is quite good. For this reason, it is common
to approximate $\sin(x)$ with its linear approximation $\l(x) = x$
when $x$ is near zero.  
%see Figure~\ref{figure:la sin}.
\end{explanation}
\end{example}





\end{document}