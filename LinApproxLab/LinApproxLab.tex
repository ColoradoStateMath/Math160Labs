\documentclass[handout,nooutcomes]{ximera}
%handout
%wordchoicegiven
%space
%nooutcomes
\title{Math 160 Lab 2}
\author{Ben Sencindiver} %Used Bart Snapp and Jim Fowler's mooculus textbook as a guide
%irrelevantchange
%\usepackage{todonotes}

\newcommand{\todo}{}

\usepackage{tkz-euclide}
\tikzset{>=stealth} %% cool arrow head
\tikzset{shorten <>/.style={ shorten >=#1, shorten <=#1 } } %% allows shorter vectors

\usetikzlibrary{backgrounds} %% for boxes around graphs
\usetikzlibrary{shapes,positioning}  %% Clouds and stars
\usetikzlibrary{matrix} %% for matrix
\usetkzobj{all}
\usepackage[makeroom]{cancel} %% for strike outs
%\usepackage{mathtools} %% for pretty underbrace % Breaks Ximera
\usepackage{multicol}


\newcommand{\RR}{\mathbb R}
%\renewcommand{\d}{\,d\!}
\renewcommand{\d}{\mathop{}\!d\!}
\newcommand{\dd}[2][]{\frac{\d #1}{\d #2}}
\renewcommand{\l}{\ell}
\newcommand{\ddx}{\frac{d}{\d x}}

\newcommand{\zeroOverZero}{\ensuremath{\boldsymbol{\tfrac{0}{0}}}}
\newcommand{\inftyOverInfty}{\ensuremath{\boldsymbol{\tfrac{\infty}{\infty}}}}
\newcommand{\zeroOverInfty}{\ensuremath{\boldsymbol{\tfrac{0}{\infty}}}}
\newcommand{\zeroTimesInfty}{\ensuremath{\small\boldsymbol{0\cdot \infty}}}
\newcommand{\inftyMinusInfty}{\ensuremath{\small\boldsymbol{\infty - \infty}}}
\newcommand{\oneToInfty}{\ensuremath{\boldsymbol{1^\infty}}}
\newcommand{\zeroToZero}{\ensuremath{\boldsymbol{0^0}}}
\newcommand{\inftyToZero}{\ensuremath{\boldsymbol{\infty^0}}}


\newcommand{\numOverZero}{\ensuremath{\boldsymbol{\tfrac{\#}{0}}}}
\newcommand{\dfn}{\textbf}
%\newcommand{\unit}{\,\mathrm}
\newcommand{\unit}{\mathop{}\!\mathrm}
\newcommand{\eval}[1]{\bigg[ #1 \bigg]}
\renewcommand{\epsilon}{\varepsilon}
\renewcommand{\iff}{\Leftrightarrow}

\DeclareMathOperator{\arccot}{arccot}
\DeclareMathOperator{\arcsec}{arcsec}
\DeclareMathOperator{\arccsc}{arccsc}
\DeclareMathOperator{\si}{Si}


\colorlet{textColor}{black} 
\colorlet{background}{white}
\colorlet{penColor}{blue!50!black} % Color of a curve in a plot
\colorlet{penColor2}{red!50!black}% Color of a curve in a plot
\colorlet{penColor3}{red!50!blue} % Color of a curve in a plot
\colorlet{penColor4}{green!50!black} % Color of a curve in a plot
\colorlet{penColor5}{orange!80!black} % Color of a curve in a plot
\colorlet{fill1}{penColor!20} % Color of fill in a plot
\colorlet{fill2}{penColor2!20} % Color of fill in a plot
\colorlet{fillp}{fill1} % Color of positive area
\colorlet{filln}{penColor2!20} % Color of negative area
\colorlet{fill3}{penColor3!20} % Fill
\colorlet{fill4}{penColor4!20} % Fill
\colorlet{fill5}{penColor5!20} % Fill
\colorlet{gridColor}{gray!50} % Color of grid in a plot

\pgfmathdeclarefunction{gauss}{2}{% gives gaussian
  \pgfmathparse{1/(#2*sqrt(2*pi))*exp(-((x-#1)^2)/(2*#2^2))}%
}



\newcommand{\fullwidth}{}
\newcommand{\normalwidth}{}


%% makes a snazzy t-chart for evaluating functions
%\newenvironment{tchart}{\rowcolors{2}{}{background!90!textColor}\array}{\endarray}

%%This is to help with formatting on future title pages.
\newenvironment{sectionOutcomes}{}{} 


\outcome{Define linear approximation as an application of the tangent to a curve.}
\outcome{Find the linear approximation to a function at a point and use it to approximate the function value.}
\outcome{Identify when a linear approximation can be used.}
\outcome{Label a graph with the appropriate quantities used in linear approximation.}
\outcome{Find the error of a linear approximation.}

\begin{document}

\section{Math 160 Lab 2 \\ Linear Approximation}

\begin{abstract}
This is Lab 2 for Math 160 - Due Wednesday, 12 October, 2016. This lab will cover tolerance and provide an introduction to the idea of limits. Unless stated otherwise, compute all values to $6$ decimal places.

We use a method called ``linear approximation'' to estimate the value of (complicated) functions at a given point.
\end{abstract}

\maketitle

\begin{example}
Suppose you want to approximate the function $f(x) = (x-2)^2  + 4$ at $x=3$.

Given a line with slope $m$ through the point $(a, b)$, what is the point-slope form of the line?
\[
y = \answer[given]{m(x - a) + b}
\]

What is the slope of the line tangent to $f(x)$ at $x=3$?
\[
\answer{2}
\]

\begin{problem}
What point should an approximation of $f(x)$ at $x=3$ go through?

\[
(\answer[given]{3}, \answer[given]{5})
\]
\begin{hint}
Should the linear approximation go through the point at which it is approximating?
\end{hint}
\end{problem}

\begin{problem}
What is the equation of the line tangent to $f(x)$ at $x-2$?
\[
y = \answer[given]{2}x + \answer[given]{-1}
\]
\begin{hint}
Think $y=mx+b$
\end{hint}

\begin{feedback}
Your line should look like this.
\[
\graph{f(x) = (x-2)^2+4, y = 2(x-3) + 5}
\]
\end{feedback}
\end{problem}




%\section{Linear approximation}

Given a function, a \textit{linear approximation} (at $x=x_0$) is a fancy phrase
for something you already know:
\begin{center}
\begin{quote}
  \textbf{The line tangent to the function (at $x=x_0$).}
\end{quote}
\end{center}
%Except in this section, the emphasis is on the \textbf{line}.


\begin{definition}\index{linear approximation}
If $f$ is a differentiable function at $x=a$, then a \textbf{linear
  approximation} for $f$ at $x=a$ is given by
\[
\l(x) = f'(a)(x-a) +f(a).
\]
\end{definition}


Note that $\l(x)$ is just the line tangent to $f(x)$ at $x=a$.


\begin{example}
Below is the graph of $f(x) = 1/8x^3 + (3/8)x^2 -3/4 - 2$ and the line
tangent to $f(x)$ at $x=a$. Feel free to explore how the tangent line
(in purple) changes as $a$ varies by using the slider. You may need to
hit `-' (zoom out) in order to see the linear approximations for large $a$ values.
\[
\graph[panel]{a=1, f(x) = 1/8x^3 + (3/8)x^2 -3/4 - 2, t(x) = f'(a)(x-a)+ f(a)}
\]

Notice how the linear approximation (the purple line) hugs
$f(x)$ (the green line) at the $x=a$ value!
\end{example}

A linear approximation of $f$ is a ``good'' approximation as long as
$x$ is ``not too far'' from $a$.
%As we see from Figure~\ref{figure:informal-tangent}, 
If one ``zooms in'' on $f$ at $x=a$ sufficiently, then $f$ and the linear
approximation are nearly indistinguishable. As a first example, we
will see how linear approximations allow us to make approximate
``difficult'' computations.
\end{example}





\end{document}