\documentclass[handout,nooutcomes]{ximera}
%handout
%wordchoicegiven
%space
%nooutcomes
\title{Math 160 Lab 2}
\author{Ben Sencindiver} %Used Bart Snapp and Jim Fowler's mooculus textbook as a guide
%irrelevantchange
\input{preamble.tex}

\outcome{Define linear approximation as an application of the tangent to a curve.}
\outcome{Find the linear approximation to a function at a point and use it to approximate the function value.}
\outcome{Identify when a linear approximation can be used.}
\outcome{Label a graph with the appropriate quantities used in linear approximation.}
\outcome{Find the error of a linear approximation.}

\begin{document}

\section{Math 160 Lab 2 \\ Linear Approximation}

\begin{example}
Boop
\begin{explanation}
Explanation boop


\begin{question}
Since we are simulating the motion of the pendulum when letting go of the pendulum from
a stand still, which of the functions you picked above would accurately simulate this situation?
\begin{selectAll}
\choice{$\sin(t)$}
\choice[correct]{$\cos(t)$}
\choice{$\tan(t)$}
\choice{$\csc(t)$}
\choice{$\csc(t)$}
\choice{$\csc(t)$}
\end{selectAll}
\begin{hint}
We want that the pendulum is not moving at $t=0$.
\end{hint}
\begin{hint}
We want a function $f(t)$ such that $f'(t) = 0$ at $t=0$.
\end{hint}
\end{question}

We're almost there! We know what the function generally looks like, but we
need to check the fine details to make sure we answer the question we set out to answer.

We want to solve the differential equation

\[
\dfrac{d^2 \theta}{d t^2}  = -\dfrac{g}{L} \theta
\]

or for the constant $C=\frac{g}{L}$,

\[
\dfrac{d^2 \theta}{d t^2}  = -C\theta.
\]

For $\theta = \cos(ax+b)$

\begin{equation*}
\begin{split}
    \dfrac{d^2}{d t^2}(\theta) &= \dfrac{d^2}{d t^2}\Big(\cos(a x + b)\Big)\\
    &= \dfrac{d}{d t} \Big( \answer[given]{- a \sin(a x+b)}\Big)\\
    &= \answer[given]{-a^2 \cos(a x+b)}\\
    &= -a^2 \theta 
\end{split}
\end{equation*}

\begin{question}
So if we are looking for a solution to $\dfrac{d^2 \theta}{d t^2}  = -\dfrac{g}{L} \theta$,
then solutions to this differential equation is $\theta = \cos(ax+b)$ where

\[
b = \answer[given]{0}
\]

and 

\[
a^2 = \answer[given]{g/L}.
\]

Assuming $a>0$, then 

\[
a = \dfrac{\answer[given]{\sqrt{g}}}{\answer[given]{\sqrt{L}}}
\]
\end{question}

\textbf{Question.}  Given your solution to the differential equation above, what is the period of a ball with mass $m$ on a string with length $L$ which is displaced at "small" angles?

\[
(\answer[given]{2\pi})\sqrt{\Big(\answer[given]{L/g}\Big)}
\]
\begin{expandable}
For $A\sin(ax+b)$ or $A\cos(ax+b)$, the period of the function is $\dfrac{2\pi}{a}$.
\end{expandable}

If the ball starts at an initial angle of displacement $\theta_0$
(where $|\theta_0|<\frac{\pi}{12}= 15^\circ$), we can describe
the position of the ball at any given time by

\[
\theta(t) = \theta_0 \cos\Big(\sqrt{\frac{g}{L}} t\Big).
\]
Pretty neat, huh?


\textbf{Question. }  Given a ball with mass $0.6$ kilograms hanging from a string with length $7$ meters, what is the period of a pendulum swing when the swing doesn't deviate more than $\frac{\pi}{12}$ radians from a free hang (where gravitational constant $9.8$ meters per second)?


The period is approximately $\answer[tolerance=0.01]{2*3.1415*0.84515}$.


\end{explanation}
\end{example}



%%%%%%%%%%%%%%%%%%%%%%%%%%%%%%%%%%%

\textbf{TESTING GROUND!}






\begin{example}
\begin{explanation}
\textbf{Question.}  Which of the following trigonometric functions have this property
($\dfrac{d^2 \theta}{d t^2} = -\theta$)? Select all such functions.
\begin{selectAll}
\choice[correct]{$\sin(t)$}
\choice[correct]{$\cos(t)$}
\choice{$\tan(t)$}
\choice{$\csc(t)$}
\choice{$\sec(t)$}
\choice{$\cot(t)$}
\end{selectAll}


\begin{question}
Since we are simulating the motion of the pendulum when letting go of the pendulum from
a stand still, which of the functions you picked above would accurately simulate this situation?
\begin{selectAll}
\choice{$\sin(t)$}
\choice[correct]{$\cos(t)$}
\choice{$\tan(t)$}
\choice{$\csc(t)$}
\choice{$\csc(t)$}
\choice{$\csc(t)$}
\end{selectAll}
\begin{hint}
We want that the pendulum is not moving at $t=0$.
\end{hint}
\begin{hint}
We want a function $f(t)$ such that $f'(t) = 0$ at $t=0$.
\end{hint}
\end{question}

We're almost there! We know what the function generally looks like, but we
need to check the fine details to make sure we answer the question we set out to answer.

We want to solve the differential equation

\[
\dfrac{d^2 \theta}{d t^2}  = -\dfrac{g}{L} \theta
\]

or for the constant $C=\frac{g}{L}$,

\[
\dfrac{d^2 \theta}{d t^2}  = -C\theta.
\]

For $\theta = \cos(ax+b)$

\begin{align*}
    \dfrac{d^2}{d t^2}(\theta) &= \dfrac{d^2}{d t^2}\Big(\cos(a x + b)\Big)\\
    &= \dfrac{d}{d t} \Big( \answer[given]{- a \sin(a x+b)}\Big)\\
    &= \answer[given]{-a^2 \cos(a x+b)}\\
    &= -a^2 \theta 
\end{align*}

\begin{question}
So if we are looking for a solution to $\dfrac{d^2 \theta}{d t^2}  = -\dfrac{g}{L} \theta$,
then solutions to this differential equation is $\theta = \cos(ax+b)$ where

\[
b = \answer[given]{0}
\]

and 

\[
a^2 = \answer[given]{g/L}.
\]

Assuming $a>0$, then 

\[
a = \dfrac{\answer[given]{\sqrt{g}}}{\answer[given]{\sqrt{L}}}
\]
\end{question}

\begin{question}
Given your solution to the differential equation above, what is the period of a ball with mass $m$ on a string with length $L$ which is displaced at "small" angles?

\[
(\answer[given]{2\pi})\sqrt{\Big(\answer[given]{L/g}\Big)}
\]
\begin{hint}
For $A\sin(ax+b)$ or $A\cos(ax+b)$, the period of the function is $\dfrac{2\pi}{a}$.
\end{hint}
\end{question}

If the ball starts at an initial angle of displacement $\theta_0$
(where $|\theta_0|<\frac{\pi}{12}= 15^\circ$), we can describe
the position of the ball at any given time by

\[
\theta(t) = \theta_0 \cos\Big(\sqrt{\frac{g}{L}} t\Big).
\]
Pretty neat, huh?

\begin{question}
Given a ball with mass $0.6$ kilograms hanging from a string with length $7$ meters, what is the period of a pendulum swing when the swing doesn't deviate more than $\frac{\pi}{12}$ radians from a free hang (where gravitational constant $9.8$ meters per second)?


The period is approximately $\answer[tolerance=0.01]{2*3.1415*0.84515}$.

\end{question}

\end{explanation}
\end{example}







\end{document}