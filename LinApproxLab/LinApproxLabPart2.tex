\documentclass[handout,nooutcomes]{ximera}
%handout
%wordchoicegiven
%space
%nooutcomes
\title{Math 160 Lab 2: Application}
\author{Ben Sencindiver} %Used Bart Snapp and Jim Fowler's mooculus textbook as a guide
%irrelevantchange
%\usepackage{todonotes}

\newcommand{\todo}{}

\usepackage{tkz-euclide}
\tikzset{>=stealth} %% cool arrow head
\tikzset{shorten <>/.style={ shorten >=#1, shorten <=#1 } } %% allows shorter vectors

\usetikzlibrary{backgrounds} %% for boxes around graphs
\usetikzlibrary{shapes,positioning}  %% Clouds and stars
\usetikzlibrary{matrix} %% for matrix
\usetkzobj{all}
\usepackage[makeroom]{cancel} %% for strike outs
%\usepackage{mathtools} %% for pretty underbrace % Breaks Ximera
\usepackage{multicol}


\newcommand{\RR}{\mathbb R}
%\renewcommand{\d}{\,d\!}
\renewcommand{\d}{\mathop{}\!d\!}
\newcommand{\dd}[2][]{\frac{\d #1}{\d #2}}
\renewcommand{\l}{\ell}
\newcommand{\ddx}{\frac{d}{\d x}}

\newcommand{\zeroOverZero}{\ensuremath{\boldsymbol{\tfrac{0}{0}}}}
\newcommand{\inftyOverInfty}{\ensuremath{\boldsymbol{\tfrac{\infty}{\infty}}}}
\newcommand{\zeroOverInfty}{\ensuremath{\boldsymbol{\tfrac{0}{\infty}}}}
\newcommand{\zeroTimesInfty}{\ensuremath{\small\boldsymbol{0\cdot \infty}}}
\newcommand{\inftyMinusInfty}{\ensuremath{\small\boldsymbol{\infty - \infty}}}
\newcommand{\oneToInfty}{\ensuremath{\boldsymbol{1^\infty}}}
\newcommand{\zeroToZero}{\ensuremath{\boldsymbol{0^0}}}
\newcommand{\inftyToZero}{\ensuremath{\boldsymbol{\infty^0}}}


\newcommand{\numOverZero}{\ensuremath{\boldsymbol{\tfrac{\#}{0}}}}
\newcommand{\dfn}{\textbf}
%\newcommand{\unit}{\,\mathrm}
\newcommand{\unit}{\mathop{}\!\mathrm}
\newcommand{\eval}[1]{\bigg[ #1 \bigg]}
\renewcommand{\epsilon}{\varepsilon}
\renewcommand{\iff}{\Leftrightarrow}

\DeclareMathOperator{\arccot}{arccot}
\DeclareMathOperator{\arcsec}{arcsec}
\DeclareMathOperator{\arccsc}{arccsc}
\DeclareMathOperator{\si}{Si}


\colorlet{textColor}{black} 
\colorlet{background}{white}
\colorlet{penColor}{blue!50!black} % Color of a curve in a plot
\colorlet{penColor2}{red!50!black}% Color of a curve in a plot
\colorlet{penColor3}{red!50!blue} % Color of a curve in a plot
\colorlet{penColor4}{green!50!black} % Color of a curve in a plot
\colorlet{penColor5}{orange!80!black} % Color of a curve in a plot
\colorlet{fill1}{penColor!20} % Color of fill in a plot
\colorlet{fill2}{penColor2!20} % Color of fill in a plot
\colorlet{fillp}{fill1} % Color of positive area
\colorlet{filln}{penColor2!20} % Color of negative area
\colorlet{fill3}{penColor3!20} % Fill
\colorlet{fill4}{penColor4!20} % Fill
\colorlet{fill5}{penColor5!20} % Fill
\colorlet{gridColor}{gray!50} % Color of grid in a plot

\pgfmathdeclarefunction{gauss}{2}{% gives gaussian
  \pgfmathparse{1/(#2*sqrt(2*pi))*exp(-((x-#1)^2)/(2*#2^2))}%
}



\newcommand{\fullwidth}{}
\newcommand{\normalwidth}{}


%% makes a snazzy t-chart for evaluating functions
%\newenvironment{tchart}{\rowcolors{2}{}{background!90!textColor}\array}{\endarray}

%%This is to help with formatting on future title pages.
\newenvironment{sectionOutcomes}{}{} 


\outcome{Define linear approximation as an application of the tangent to a curve.}
\outcome{Find the linear approximation to a function at a point and use it to approximate the function value.}
\outcome{Identify when a linear approximation can be used.}
\outcome{Label a graph with the appropriate quantities used in linear approximation.}
\outcome{Find the error of a linear approximation.}

\begin{document}

\section{Math 160 Lab 2 \\ Linear Approximation Application}

\begin{abstract}
This section of the lab walk through an application using linear approximation.
\end{abstract}

\maketitle


%%%%%%%%%%%%%%%%%%%%%%%%%%%%%%%%%%%%%%%%

%%% Approximating periodicity of a pendulum!
Let's apply this notion of linear approximations in a physics application to
answer a real-life question!


\begin{example}
Consider a pendulum with ball $B$ with mass $m$ hanging from a string
with length $L$ hanging from a block of woods at pivot point $P$.
Below is a picture of the pendulum that is displaced with angle
$\theta$ from equilibrium.

%% Picture of pendulum 
%Edited from http://tex.stackexchange.com/questions/219038
%/tikz-draw-angle-with-label-between-lines
\begin{center}
\usetikzlibrary{calc,patterns,angles,quotes}
\begin{tikzpicture}
    \coordinate (origo) at (0,0);
    \coordinate (pivot) at (1,5);

    % draw axes
    \fill[black] (origo) circle (0.05);
    \node[below left=1pt of {(0,0)}] {Point $P$};
    \draw[thick,gray,->] (origo) -- ++(4,0) node[black,right] {};
    \draw[thick,dashed, gray,->] (origo) -- ++(0,-4) node (mary) [black,below] {};

    % draw roof
    \fill[pattern = north east lines] ($ (origo) + (-1,0) $) rectangle ($ (origo) + (1,0.5) $);
    \draw[thick] ($ (origo) + (-1,0) $) -- ($ (origo) + (1,0) $);

    \draw[thick] (origo) -- ++(300:3) coordinate (bob);
    \fill (bob) circle (0.2);
    \node[below left=5pt of {(1.5,-2.5)}] {$B$};

    \pic [draw, ->, "$\theta$", angle eccentricity=1.5] {angle = mary--origo--bob};
  \end{tikzpicture}
\end{center}

In this model when the mass is directly below the pivot point $P$
(on the dashed line), the string is in equilibrium and doesn't move.

Suppose we'd like to find out how long it takes this pendulum
to swing back and forth (one period).

\begin{explanation}
First of all to find the period of the pendulum, we need to understand
how the pendulum moves. Note that the position of the pendulum is completely determined
by the angle of displacement $\theta$ (Assuming that the string is always tight).
If we move the ball of the pendulum to a certain $\theta\neq 0$ and let go,
the pendulum will swing back and forth. Therefore the angle $\theta$
is a function of time, call it $t$. So how does this bad-boy move?

Newton's second law of motion implies the following 
relationship holds in this system:

\[
-m g L \phantom{.} \sin(\theta) = m L^2 \phantom{.}\dfrac{d^2 \theta}{d t^2}
\]
where $g$ is earth's gravitational constant.

Our goal is to solve this differential equation to find what the function
$\theta$ is. Once we find $\theta$, we can find the period of the pendulum's swing.

Let's get to work. Manipulate the above equation so that `$0$'
is on one side and the $\dfrac{d^2 \theta}{d t^2}$ term is positive.

\[
0 = \answer[given]{m L^2} \dfrac{d^2 \theta}{d t^2}  + \answer[given]{m g L \sin(\theta)}
\]

Note that $L,m,$ and $g$ are constant and greater than $0$, so we can
divide the equality by any of these terms to get an equivalent statement.
Manipulate the above equality so that the coefficient of $\dfrac{d^2 \theta}{d t^2}$
is $1$. Be sure to simplify fractions.

\[
0 = \dfrac{d^2 \theta}{d t^2}  + \dfrac{\answer[given]{g}}{\answer[given]{L}} \phantom{.} \sin(\theta)
\]
\end{explanation}
\end{example}

\begin{problem}
Does the above relationship depend on the length of the string?
\begin{multipleChoice}
\choice[correct]{Yes}
\choice{No}
\end{multipleChoice}
\begin{feedback}
There's no way to eliminate $L$ from the equation, so the solution
to the differential equation depends on the length of the string $L$.
\end{feedback}
\end{problem}

\begin{problem}
Does the above relationship depend on the gravitational constant?
\begin{multipleChoice}
\choice[correct]{Yes}
\choice{No}
\end{multipleChoice}
\begin{feedback}
There's no way to eliminate $g$ from the equation, so the solution
to the differential equation depends on earth's gravitational
constant $g$.
\end{feedback}
\end{problem}

\begin{problem}
Does the above relationship depend on the mass of ball?
\begin{multipleChoice}
\choice{Yes}
\choice[correct]{No}
\end{multipleChoice}
\begin{feedback}
By dividing by $m$, we were able to eliminate the mass $m$ from
the equation. Therefore the solution to the differential equation
(the position/path of the ball) doesn't depend
on the mass of the ball. Even though $m$ was in the initial equality,
it isn't relevant to the solution. Holy smokes!
\end{feedback}
\end{problem}

\begin{example}
\begin{explanation}
Now let's try to solve this differential equation!

\[
0 = \dfrac{d^2 \theta}{d t^2}  + \dfrac{g}{L} \sin(\theta)
\]

We will utilize the fact that $\theta$, the angle of displacement of the ball $B$),
is a function of time, $t$. We are trying to solve the differential
equation above so that we can find the period of the swing of the pendulum.

Manipulate the above equation to so that $\dfrac{d^2 \theta}{d t^2}$ is in terms of $\theta$.

\[
\dfrac{d^2 \theta}{d t^2}  = \answer[given]{-g\sin(\theta)/L}
\]

The differential equation is too tough for us to solve at the moment,
but we can solve something similar.


Let's solve this differential equation for small displacements
of $\theta$. We will replace `$\sin(\theta)$' with a linear approximation
of $\sin(\theta)$.



\textbf{Question. } If we want to solve this differential displacements from equilibrium, at value of $\theta$ should we approximate $\sin(\theta)$?


We find a linear approximation of $\sin(\theta)$ at $\theta = \answer[given]{0}$.\\



\textbf{Question. } What is the linear approximation of $\sin(\theta)$ at the above $\theta$ value?

\[
\sin(\theta) \approx \answer[given]{\theta}
\]

\begin{expandable}
Hint: Remember $\theta$ is the variable, not $t$.
\begin{expandable}
You've already found this early this earlier in this lab.
\end{expandable}
\end{expandable}



\textbf{Remark. } The above approximation is actually a surprisingly good for
$|\theta|\leq \frac{\pi}{12} = 15^\circ$, so in this context, "small" angles
are angles that deviate no more that $15^\circ$ from equilibrium, FYI.\\


Therefore we are trying to find a solution to

\[
\dfrac{d^2 \theta}{d t^2}  = -\dfrac{g}{L} \theta
\]

or for the constant $C=\frac{g}{L}$,

\[
\dfrac{d^2 \theta}{d t^2}  = -C \theta.
\]

Another way to phrase this is that $\theta$ is a functions whose
second derivative is $-1$ times a scalar $C$ of the original function.


\textbf{Question.}  Do you know any types of functions whose \emph{second} derivative
is $-1$ times itself (disregarding the $C$)? Which of the following have at least one function that has this property.

\begin{selectAll}
\choice[correct]{Polynomials of degree $0$ (Constant functions)}
\choice{Polynomials of degree $1$ or bigger}
\choice[correct]{Trigonometric functions}
\end{selectAll}



\textbf{Problem. } Which polynomial $f(t)$ has the property that it's second
derivative is $-1$ times itself i.e. if
$\theta=f(t)$, then $\dfrac{d^2 \theta}{d t^2} = f''(t) = -f(t)$?


The polynomial function $f(t) = \answer[given]{0}$ has the property that $f''(t) = -f(t)$.

\begin{expandable}
Don't over think it.
\begin{expandable}
It is polynomial in the variable $t$.
\begin{expandable}
It is actually a constant function.
\end{expandable}
\end{expandable}
\end{expandable}




This is our equilibrium solution! This function has $\theta=0$ at $t=0$. In other
words, this is our solution when we put the ball directly below the point
that the string is hanging from. This is no surprise though since this
is the equilibrium. What we \emph{really} want is to find out what happens
when we place the ball somewhere where the ball actually swings.


\textbf{Question.}  Which of the following trigonometric functions have this property
($\dfrac{d^2 \theta}{d t^2} = -\theta$)? Select all such functions.
\begin{selectAll}
\choice[correct]{$\sin(t)$}
\choice[correct]{$\cos(t)$}
\choice{$\tan(t)$}
\choice{$\csc(t)$}
\choice{$\sec(t)$}
\choice{$\cot(t)$}
\end{selectAll}



\textbf{Question }. Since we are simulating the motion of the pendulum when letting go of the pendulum from
a stand still, which of the functions you picked above would accurately simulate this situation?
\begin{selectAll}
\choice{$\sin(t)$}
\choice[correct]{$\cos(t)$}
\choice{$\tan(t)$}
\choice{$\csc(t)$}
\choice{$\csc(t)$}
\choice{$\csc(t)$}
\end{selectAll}
\begin{expandable}
We want that the pendulum is not moving at $t=0$.
\begin{expandable}
We want a function $f(t)$ such that $f'(t) = 0$ at $t=0$.
\end{expandable}
\end{expandable}



We're almost there! We know what the function generally looks like, but we
need to check the fine details to make sure we answer the question we set out to answer.

We want to solve the differential equation

\[
\dfrac{d^2 \theta}{d t^2}  = -\dfrac{g}{L} \theta
\]

or for the constant $C=\frac{g}{L}$,

\[
\dfrac{d^2 \theta}{d t^2}  = -C\theta.
\]

For $\theta = \cos(at+b)$

\begin{align*}
    \dfrac{d^2}{d t^2}(\theta) &= \dfrac{d^2}{d t^2}\Big(\cos(a t + b)\Big)\\
    &= \dfrac{d}{d t} \Big( \answer[given]{- a \sin(a t+b)}\Big)\\
    &= \answer[given]{-a^2 \cos(a t+b)}\\
    &= -a^2 \theta 
\end{align*}


\textbf{Question. } So if we are looking for a solution to $\dfrac{d^2 \theta}{d t^2}  = -\dfrac{g}{L} \theta$,
then solutions to this differential equation is $\theta = \cos(at+b)$ where

\[
b = \answer[given]{0}
\]

and 

\[
a^2 = \answer[given]{g/L}.
\]

Assuming $a>0$, then 

\[
a = \dfrac{\answer[given]{\sqrt{g}}}{\answer[given]{\sqrt{L}}}
\]



\textbf{Question. } Given your solution to the differential equation above, what is the period of a ball with mass $m$ on a string with length $L$ which is displaced at "small" angles?

\[
(\answer[given]{2\pi})\sqrt{\Big(\answer[given]{L/g}\Big)}
\]
\begin{expandable}
Hint: For $A\sin(at+b)$ or $A\cos(at+b)$, the period of the function is $\dfrac{2\pi}{a}$.
\end{expandable}


If the ball starts at an initial angle of displacement $\theta_0$
(where $|\theta_0|<\frac{\pi}{12}= 15^\circ$), we can describe
the position of the ball at any given time by

\[
\theta(t) = \theta_0 \cos\Big(\sqrt{\frac{g}{L}} t\Big).
\]
Pretty neat, huh?


\textbf{Question. } Given a ball with mass $0.6$ kilograms hanging from a string with length $7$ meters, what is the period of a pendulum swing when the swing doesn't deviate more than $\frac{\pi}{12}$ radians from a free hang (where gravitational constant $9.8$ meters per second)?


The period is approximately $\answer[tolerance=0.01]{2*3.1415*0.84515}$.

%http://www.acs.psu.edu/drussell/Demos/Pendulum/Pendula.html
\end{explanation}
\end{example}







\end{document}