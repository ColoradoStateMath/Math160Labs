\documentclass[handout,nooutcomes]{ximera}
%handout
%wordchoicegiven
%space
%nooutcomes
\title{Math 160 Lab 3}
\author{Jeremy Buss} %Used Bart Snapp and Jim Fowler's mooculus textbook, and Ben Sencindiver's Linear Approx. Lab as a guide
%\usepackage{todonotes}

\newcommand{\todo}{}

\usepackage{tkz-euclide}
\tikzset{>=stealth} %% cool arrow head
\tikzset{shorten <>/.style={ shorten >=#1, shorten <=#1 } } %% allows shorter vectors

\usetikzlibrary{backgrounds} %% for boxes around graphs
\usetikzlibrary{shapes,positioning}  %% Clouds and stars
\usetikzlibrary{matrix} %% for matrix
\usetkzobj{all}
\usepackage[makeroom]{cancel} %% for strike outs
%\usepackage{mathtools} %% for pretty underbrace % Breaks Ximera
\usepackage{multicol}


\newcommand{\RR}{\mathbb R}
%\renewcommand{\d}{\,d\!}
\renewcommand{\d}{\mathop{}\!d\!}
\newcommand{\dd}[2][]{\frac{\d #1}{\d #2}}
\renewcommand{\l}{\ell}
\newcommand{\ddx}{\frac{d}{\d x}}

\newcommand{\zeroOverZero}{\ensuremath{\boldsymbol{\tfrac{0}{0}}}}
\newcommand{\inftyOverInfty}{\ensuremath{\boldsymbol{\tfrac{\infty}{\infty}}}}
\newcommand{\zeroOverInfty}{\ensuremath{\boldsymbol{\tfrac{0}{\infty}}}}
\newcommand{\zeroTimesInfty}{\ensuremath{\small\boldsymbol{0\cdot \infty}}}
\newcommand{\inftyMinusInfty}{\ensuremath{\small\boldsymbol{\infty - \infty}}}
\newcommand{\oneToInfty}{\ensuremath{\boldsymbol{1^\infty}}}
\newcommand{\zeroToZero}{\ensuremath{\boldsymbol{0^0}}}
\newcommand{\inftyToZero}{\ensuremath{\boldsymbol{\infty^0}}}


\newcommand{\numOverZero}{\ensuremath{\boldsymbol{\tfrac{\#}{0}}}}
\newcommand{\dfn}{\textbf}
%\newcommand{\unit}{\,\mathrm}
\newcommand{\unit}{\mathop{}\!\mathrm}
\newcommand{\eval}[1]{\bigg[ #1 \bigg]}
\renewcommand{\epsilon}{\varepsilon}
\renewcommand{\iff}{\Leftrightarrow}

\DeclareMathOperator{\arccot}{arccot}
\DeclareMathOperator{\arcsec}{arcsec}
\DeclareMathOperator{\arccsc}{arccsc}
\DeclareMathOperator{\si}{Si}


\colorlet{textColor}{black} 
\colorlet{background}{white}
\colorlet{penColor}{blue!50!black} % Color of a curve in a plot
\colorlet{penColor2}{red!50!black}% Color of a curve in a plot
\colorlet{penColor3}{red!50!blue} % Color of a curve in a plot
\colorlet{penColor4}{green!50!black} % Color of a curve in a plot
\colorlet{penColor5}{orange!80!black} % Color of a curve in a plot
\colorlet{fill1}{penColor!20} % Color of fill in a plot
\colorlet{fill2}{penColor2!20} % Color of fill in a plot
\colorlet{fillp}{fill1} % Color of positive area
\colorlet{filln}{penColor2!20} % Color of negative area
\colorlet{fill3}{penColor3!20} % Fill
\colorlet{fill4}{penColor4!20} % Fill
\colorlet{fill5}{penColor5!20} % Fill
\colorlet{gridColor}{gray!50} % Color of grid in a plot

\pgfmathdeclarefunction{gauss}{2}{% gives gaussian
  \pgfmathparse{1/(#2*sqrt(2*pi))*exp(-((x-#1)^2)/(2*#2^2))}%
}



\newcommand{\fullwidth}{}
\newcommand{\normalwidth}{}


%% makes a snazzy t-chart for evaluating functions
%\newenvironment{tchart}{\rowcolors{2}{}{background!90!textColor}\array}{\endarray}

%%This is to help with formatting on future title pages.
\newenvironment{sectionOutcomes}{}{} 


\outcome{Apply the notion of extreme values to find an optimal solution to a problem.}
\outcome{Explore a novel application of optimization techniques.}
\outcome{Interpret the results of an optimal solution in context.}

\begin{document}

\section{Math 160 Lab 3 \\ An Optimal Shape}

%% Have to edit the date here each semester.
\begin{abstract}
This is Lab 3 for Math 160 - Due Wednesday, March 29, 2017 at 5:00PM MDT.
This lab will guide you through an optimization problem whose solution is relevant in many fields of study.\\

Unless stated otherwise, input answers in \underline{exact form} in this lab.
\end{abstract}

\maketitle

\hspace{2cm}In class we explored a problem that involved finding the optimal dimensions for a rectangular pen that maximizes its area. Recall that the solution when all four sides of the pen will be built from a limited ammount of fencing is a square. In this lab we will examine what happens if we are not restricted to a rectangle, but build the pen as a regular polygon with n sides. Our goal is finding the optimal number of sides to use when building a pen.\\

\medskip
%%%%%% Area of an n-sided regular polygon
\hspace{2cm}To optimize the area of an n-sided regular polygon we will need a function that represents the area of the polygon. One way to do this is to split the polygon into smaller, more manageable pieces. Since we are working with a regular polygon (A polygon whose sides and angles are all the same size), a natural choice is to cut it into triangles from the center as in the figure below.

%%Fancy nice picture of a triangle
%\begin{center}
%\usetikzlibrary{calc,patterns,angles,quotes}
%\begin{tikzpicture}
    %coordinates of octagon
%    \coordinate (origin) at (200,200);
%    \coordinate (a) at (200,0);
%    \coordinate (b) at (341,140);
%    \coordinate (c) at (400,200);
%    \coordinate (d) at (341,341);
%    \coordinate (e) at (200,400);
%    \coordinate (f) at (140,341);
%    \coordinate (g) at (0,200);
%    \coordinate (h) at (140,140);
%    \coordinate (midpoint) at (270,70)
 
    % draw octagon
%    \draw[thick] (a) -- (b);
%    \draw[thick] (b) -- (c);
%    \draw[thick] (c) -- (d);
%    \draw[thick] (d) -- (e);
%    \draw[thick] (e) -- (f);
%    \draw[thick] (f) -- (g);
%    \draw[thick] (g) -- (h);
%    \draw[thick] (h) -- (a);
    
    %draw innards
%    \draw[thick] (origin) -- (midpoint);
%    \draw[thick, gray] (origin) -- (f);
%    \draw[thick, gray] (a) -- (e);
%    \draw[thick, gray] (c) -- (g);
%    \draw[thick, gray] (d) -- (h);
    
%    \pic [draw, "$\theta$", angle eccentricity=1.5] {angle = a--origin--midpoint};
%  \end{tikzpicture}
%\end{center}

How many triangles will we split an n-sided polygon into? $\answer{n}$.\\
How large is the angle $\theta$ in the diagram? $\answer{\pi/n}$.\\
What is the area of each triangle? $\answer{r^2*\sin(\pi/n)*\cos(pi/n)}$.\\
Then what is the total area of an n-sided regular polygon? Area $=\answer{n*r^2*\sin(\pi/n)*\cos(\pi/n)}$.\\

\bigskip

\hspace{2cm}Recall when finding the optimal dimensions for the bunny pen in class we found an area formula in terms of two variables. We wished to find the derivative, so we needed to rewrite the formula in terms of only one variable. We used the fact that the length of fencing was limited in order to write one variable in terms of the other. Let's try the same strategy here.\\
Let $P$, a constant, be the length of fence that we can use to build the pen. Refer back to the triangle diagram to compute the perimeter of the polygon in terms of $n$ and $r$?\\
$P = \answer{2*r*n*\sin(\pi/n)}$.\\
Since we're keeping the perimeter $P$ constant, we can solve for $r$...\\
$r = \answer{P/(2*n*\sin(\pi/n))}$.\\
...and use it to rewrite area in terms of $n$!\\
$A(n) = \answer{P^2*\cot(\pi/n)/(4*n)}$.\\

\bigskip

What is the domain of $A(n)$? \\
\wordChoice{\choice{real numbers}\choice[correct]{integers}} \wordChoice{\choice[correct]{greater than}\choice{less than}} $\answer{3}$.\\
\begin{freeResponse}
If we find a critical point at a non-integer like $n = 4.5$ where does it make sense to look for extreme values?
\end{freeResponse}

\bigskip

\hspace{2cm}Now we're ready to find the derivative! It may be helpful to simplify your formula for the area first. Rewrite it so that it only has one trig function.\\
You will want to be able to copy and paste the derivative for later use. Write it in the text box below, or a separate text document, and paste it into the answer box for $A'(n)$ to check your work.\\

\begin{freeResponse}
We recommend making changes in the text box and pasting from there if it takes more than one try.
\end{freeResponse}
$A'(n) = P^2\,\answer{(\pi/n*\csc^2(\pi/n)-\cot(\pi/n))/n^2}$.\\

\bigskip

\hspace{2cm}Yikes! Ok, that's a beast, but let's not lose our heads. We're hunting critical points after all. Where can critical points occur?
\begin{selectAll}
    \choice[correct]{Where $A'(n)$ is zero.}
    \choice{At endpoints of the domain of $A(n)$.}
    \choice[correct]{Where $A'(n)$ does not exist.}
    \choice{Where the $A(n)$ is zero.}
    \choice[correct]{Interior points of the domain of $A(n)$.}
\end{selectAll}

\bigskip

Is there anywhere were $A'(n)$ does not exist? $\,n_0 = \answer{0}$.\\
Is $n_0$ a critical point?\\
\begin{multipleChoice}
   \choice{No, $n_0$ is not a critical point because $n_0$ is not in the domain of $A'(n)$.}
   \choice{Yes, if $A'(n_0)$ does not exist then $n_0$ is a critical point.}
   \choice{No, $n_0$ is not a critical point because $A'(n_0) \ne 0$.}
   \choice[correct]{No, $n_0$ is not a critical point because $n_0$ is not in the domain of $A(n)$.}
\end{multipleChoice}

\bigskip

\hspace{2cm}Now to see if $A'(n)=0$ anywhere. This function can't be solved algebraically, but let's persist! After all, algebra is not our only tool! Use the Desmos widget to plot a graph of $A'(n)$.\\
Replace $A'(n)=n^2$ with the derivative you computed earlier. (You'll need to leave the $P^2$ off.)
\graph[panel]{A'(n)=n^2}
Where is $A'(n)=0$?\\
\begin{multipleChoice}
    \choice{$A'(n)=0$ at $n=\infty$}
    \choice[correct]{$A'(n)$ is never zero, it is always positive}
    \choice{$A'(n)$ is never zero, it is always negative}
\end{multipleChoice}

\bigskip

So we found $\answer{0}$ critical points.\\
Can $A(n)$ have any extreme values?\\
\begin{multipleChoice}
    \choice{No, extreme values always occur at critical points.}
    \choice{No, $A(n)$ is always increasing so there cannot be any extreme values}
    \choice{Yes, $A(n)$ can have a maximum at $n=0$}
    \choice[correct]{Yes, extreme values can occur at endpoints too.}
\end{multipleChoice}

\bigskip

Since $A'(n)$ is always \wordChoice{\choice[correct]{positive}\choice{negative}}, $A(n)$ is always \wordChoice{\choice[correct]{increasing}\choice{decreasing}}.\\
At the endpoint $n=3$ $A(n)$ has a \wordChoice{\choice{local maximum}\choice[correct]{local minimum}\choice{neither a max or a min}}\\
Intuitively, do you think $A(n)$ can \wordChoice{\choice[correct]{increase}\choice{decrease}} without bound? \wordChoice{\choice{yes}\choice[correct]{no}}.\\ %%This could be interesting to look at. Does a 'yes' indicate a lack of intuition or a lack of motivation?

\medskip

\begin{exercise}
Let's investigate! Hints are provided if you need them.\\
\[\lim_{n \to \infty}A(n) = \answer{P^2/(4*\pi)}\]
\begin{hint}
	What is the limit of the numerator? $\answer{P^2}$.
\end{hint}
\begin{hint}
  Now consider the denominator. Recall \[\lim_{h \to 0} \sin(h)/h = \answer{1}\].
\end{hint}
\begin{hint}
  Can we manufacture a denominator for $sin(\pi/n)$ that matches its input?\\
  $4n\sin(\pi/n) = 4(1/(1/n))\sin(\pi/n) = 4\answer{\pi}(\sin(\pi/n)/(\answer{\pi}/n))$.
\end{hint}
\begin{hint}
  Then \[\lim_{n \to \infty} 4n\sin(\pi/n) = \lim_{n \to \infty} 4\pi(\sin(\pi/n)/(\pi/n)) = \answer{4\pi}\].
\end{hint}
\end{exercise}

What does this limit mean? Remember $A(n)$ represents
\begin{multipleChoice}
    \choice[correct]{The area of an n-sided regular polygon}
    \choice{An n-sided regular polygon}
    \choice{The perimeter of an n-sided regular polygon}
\end{multipleChoice}
\begin{freeResponse}
As $n$ approaches infinity, what happens to the shapes represented by $A(n)$?
\end{freeResponse}

What shape has maximal area for a fixed perimeter $P$?
\begin{multipleChoice}
    \choice{A triangle}
    \choice{A parabola}
    \choice{A square}
    \choice[correct]{A circle}
\end{multipleChoice}

\bigskip

If $P = 4\pi$ then \[\lim_{n \to \infty} A(n) = \answer{4\pi}\].\\
\begin{freeResponse}
Does this answer make sense with respect to the shape that you think is best? Explain.
\end{freeResponse}




\end{document}
